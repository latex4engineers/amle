\documentclass[12pt,a4paper]{report}
\usepackage[utf8]{inputenc}
\usepackage[ngerman]{babel}
\usepackage{graphicx}
\usepackage{geometry}
\geometry{left=2.00cm, right=2.00cm, top=2.00cm, bottom=2.00cm}
\pagestyle{empty}

\usepackage{gnuplottex}
\begin{document}
\begin{gnuplot}[terminal=epslatex, scale=1.2]
	set title 'Plot einer Geraden' # Diagrammtitel
	set output "plot-ausgabe-datei.tex" # Ausgabedatei
	set datafile separator ',' # Spaltenseparator
	set xrange [40:160] # Dargestellter Diagrammbereich x-Achse
	set yrange [40:160] # Dargestellter Diagrammbereich y-Achse
	# Gitter/Raster: Hauptteilung mit start, schrittweite, stop
	set xtics 40,10,160 # Hauptteilung x-Achse 
	set ytics 40,10,160 # Hauptteilung y-Achse
	# Zwischenteilung: Anzahl der Teilstriche + Hauptstrich
	set mxtics 10 # Zwischenteilung x-Achse 
	set mytics 10 # Zwischenteilung y-Achse 
	set xlabel '$x$' # Beschriftung x-Achse
	set ylabel '$g(x)$' # Beschriftung y-Achse
	set size 1.3,1 # Verhältnis Breite zu Höhe
	show grid # Rasteranzeige aktivieren
	# Formatierung des Gitters, line type, line width, line color
	set grid ytics lt 1 lw 1 lc rgb '#bbbbbb' # Formatierung x
	set grid xtics lt 1 lw 0.5 lc rgb '#bbbbbb' # Formatierung y
	set key left top Left # Position der Legende
	show key # Legende darstellen
	g(x)=x # Funktionsdefinition
	plot g(x) with lines lc rgb '#000000' lw 1 title '$g_1(x)$'
\end{gnuplot}
%\gnuplotloadfile[terminal=epslatex, scale=1.2]{gnuplotdatei.gnuplot}
\input{plot-ausgabe-datei.tex}
\end{document}

