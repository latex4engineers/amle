\documentclass[12pt,a4paper,final]{scrbook}
\usepackage[utf8]{inputenc}
\usepackage[T1]{fontenc}
\usepackage[ngerman]{babel}
\usepackage{lmodern}

\usepackage{blindtext}
\usepackage{hyperref}

\begin{filecontents}{literatur.bib}
@article{linuxcncrte,
author={Wings, Elmar and Müller, Marcel and Rochler, Marc},
year={2015},
journal={The International Journal of Advanced Manufacturing Technology},
volume={78},
number={9-12},
title={{Integration of real-time Ethernet in LinuxCNC}},
publisher={Springer London},
pages={1837-1846}
}
@book{computernetze,
author = {Patrick-Benjamin Bök and Andreas Noack and Marcel Müller and Daniel Behnke},
title = {Computernetze und Internet of Things},
year = {2020},
month = {09},
day = {11},
subtitle = {Technische Grundlagen und Spezialwissen},
publisher = {Springer Vieweg, Wiesbaden},
isbn = {978-3-658-29409-0}
}
\end{filecontents}

\usepackage[style=ieee-alphabetic]{biblatex}
\addbibresource{literatur.bib}

\begin{document}
	
\chapter{Erstes Kapitel}\label{chap:eins}
In Kapitel \ref{chap:eins} befindet sich die Einleitung. 
Die Abschnitte \ref{sec:einsA} und \ref{sec:einsB} führen in das Thema ein. 
SpringerLink\footnote{\url{https://link.springer.com/}} bietet viele interessante Fachbücher zum Download an. 
Die Datenbank IEEE Xplore\footnote{\url{https://ieeexplore.ieee.org/}} bietet viele interessante Artikel zum Download an. 
\section{Erster Abschnitt im ersten Kapitel}\label{sec:einsA}
\blindtext
\section{Zweiter Abschnitt im ersten Kapitel}\label{sec:einsB}
\blindtext

\chapter{Zweites Kapitel}\label{chap:zwei}
In Kapitel \ref{chap:zwei} wird ...
Abschnitt \ref{sec:zweiA} und Abschnitt \ref{sec:zweiB} ...
In dem Artikel \cite{linuxcncrte} ...
In dem Buch \cite{computernetze} ...
\section{Erster Abschnitt im zweiten Kapitel}\label{sec:zweiA}
\blindtext
\section{Zweiter Abschnitt im zweiten Kapitel}\label{sec:zweiB}
\blindtext

\printbibliography

\end{document}