\documentclass[12pt,a4paper]{scrbook}
\usepackage[utf8]{inputenc}
\usepackage[T1]{fontenc}
\usepackage[ngerman]{babel}
\usepackage{lmodern}

\usepackage{amsmath}
\usepackage{amsfonts}
\usepackage{amssymb}

\usepackage{graphicx}

\usepackage[left=2.50cm, right=2.50cm, top=2.00cm, bottom=2.00cm,
bindingoffset=8mm, includehead, includefoot]{geometry} % Loesung 8.2

\usepackage[onehalfspacing]{setspace} % Loesung 8.3
\setlength{\parskip}{6pt} % Loesung 8.3
\setlength{\parindent}{0pt} % Loesung 8.3

\usepackage{fancyhdr} % Loesung 8.4
\pagestyle{fancy} % Loesung 8.4
\fancyhead{} % Loesung 8.4
\fancyhead[RO]{\rightmark} % Loesung 8.4
\fancyhead[LE]{\leftmark} % Loesung 8.4
\fancyfoot{} % Loesung 8.4
\fancyfoot[RO]{\thepage} % Loesung 8.4
\fancyfoot[LE]{\thepage} % Loesung 8.4
\renewcommand{\headrulewidth}{0.4pt} % Loesung 8.4
%\renewcommand{\footrulewidth}{0.4pt} % Loesung 8.4

\usepackage{blindtext}
\usepackage{acronym}

\usepackage[style=ieee-alphabetic]{biblatex}
\addbibresource{literatur.bib}

\usepackage{makeidx}
\makeindex

\usepackage[intoc]{nomencl}
\renewcommand{\nomname}{Symbolverzeichnis}
\renewcommand{\nomlabel}[1]{#1 \dotfill}
\makenomenclature


\begin{document}
 
\begin{titlepage}
     
     \begin{flushleft} 
         \includegraphics[width=7cm]{logo.png}
     \end{flushleft} 
     
     \begin{flushright}
         \vspace{2cm}
         \LARGE \textsl{Bachelorarbeit}\\
         \rule{0.6\textwidth}{0.4pt} ~\\
         \vspace{0.5cm}
         \textsf{\LARGE Beispiel eines Layouts für}\\
         \textsf{\LARGE eine Bachelorarbeit in \LaTeX}
     \end{flushright}
     
     \vspace{3cm}
     \large
     \begin{tabbing}
         xxxxxxxxxxxxxxxx \= \kill
         Autor: \> Vorname Nachname \\
         Matrikel-Nr.: \> xxxxxxx \\
         Studiengang: \> Maschinenbau \\ [0.5cm]
         Erstprüfer: \> Prof. Dr. Ute Mustermann \\
         Zweitprüfer: \> Prof. Dr. Max Mustermann \\ [0.5cm]
         Abgabedatum: \> 12.02.2021 \\
     \end{tabbing}
     
     \vspace{0.5cm}
     \small
     \begin{center}
         Hochschule Beispiel $\cdot$ 
         Fachbereich Beispiel $\cdot$ 
         Abteilung Beispiel \\
         Musterstraße 8 $\cdot$ 
         12345 Musterstadt $\cdot$ 
         http://www.\_\_\_\_\_.de
     \end{center}
     
\end{titlepage}
\newpage

\pagenumbering{roman}

\tableofcontents
\cleardoublepage

\addcontentsline{toc}{chapter}{Abbildungsverzeichnis}
\listoffigures
\cleardoublepage

\addcontentsline{toc}{chapter}{Tabellenverzeichnis}
\listoftables
\cleardoublepage
 
\printnomenclature
\cleardoublepage
 
\nomenclature[]{$\alpha$}{Drehwinkel um die $x$-Achse}
\nomenclature[]{$\beta$}{Drehwinkel um die $y$-Achse}
\nomenclature[]{$\gamma$}{Drehwinkel um die $z$-Achse}
\nomenclature[]{$\vec{F}$}{Kraft}
\nomenclature[]{$\vec{v}$}{Geschwindigkeit}

 
\chapter*{Abkürzungen}
\addcontentsline{toc}{chapter}{Abkürzungen}
\begin{acronym}
     \acro{cnc}[CNC]{Computerized Numerical Control}
     \acro{sps}[SPS]{Speicherprogrammierbare Steuerung}
\end{acronym}
\cleardoublepage
 
\pagenumbering{arabic}

\chapter{Erstes Kapitel}
Eine \ac{sps} ist $\ldots$

Eine \ac{cnc} benötigt eine \ac{sps} \index{Speicherprogrammierbare Steuerung} (SPS) \index{SPS|see{Speicherprogrammierbare Steuerung}} zur $\ldots$

\chapter{CAGD}
... CAGD\index{CAGD} \cite{Farin2001} ...
... Splines\index{CAGD!Splines} ... \cite{DIN66025}
... \cite{Farouki2017} ...
... 3D-Drucker\index{3D-Drucker} ... \cite{Patent3D}
... 5G\index{5G}\footcite{Zafeiropoulos2020} ...

\section{Testsection}
\blindtext
\blindtext
\blindtext
 
\chapter{Letztes Kapitel}
 
\blindtext
\cleardoublepage

\addcontentsline{toc}{chapter}{Literaturverzeichnis}
\printbibliography
\cleardoublepage

\renewcommand{\indexname}{Stichwortverzeichnis}
\addcontentsline{toc}{chapter}{Stichwortverzeichnis}
\printindex
\end{document}