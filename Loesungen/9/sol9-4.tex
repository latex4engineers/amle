\documentclass[12pt,a4paper]{scrbook}
\usepackage[utf8]{inputenc}
\usepackage[T1]{fontenc}
\usepackage[ngerman]{babel}
\usepackage{xcolor}


% Darstellung von Python-Code
\newcommand{\PYTHON}[1]{\textcolor{green}{\ttfamily #1}}

% Umgebung zur Erläuterung von Python-Befehln
\newenvironment{PythonDef}%
{%
\begin{tabular}{l|l}
    \textbf{Befehl} & \textbf{Erläuterung} \\ \hline 
}%
{%
\end{tabular}%
}


\begin{document}

\chapter{Python-Befehle}

\begin{PythonDef}
 \PYTHON{import} & Laden einer Bibliothek \\ 
 \PYTHON{def}    & Definition einer Funktion \\
\end{PythonDef}
 
\bigskip
    
Es geht aber auch schöner.

% Umgebung zur Erläuterung von Python-Befehln
\renewenvironment{PythonDef}%
{%
    \renewcommand{\arraystretch}{1.5}    
    \begin{center}
        \begin{tabular}{l|l}
            \textbf{Befehl} & \textbf{Erläuterung} \\ \hline 
        }%
        {%
        \end{tabular}%
    \end{center}%
}


\begin{PythonDef}
    \PYTHON{import} & Laden einen Bibliothek \\ 
    \PYTHON{def}    & Definition einer Funktion \\
\end{PythonDef}

    
\end{document}