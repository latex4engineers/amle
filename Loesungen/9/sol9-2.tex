\documentclass[12pt,a4paper]{scrbook}
\usepackage[utf8]{inputenc}
\usepackage[T1]{fontenc}
\usepackage[ngerman]{babel}
\usepackage{xcolor}

% Darstellung von Python-Code
\newcommand{\PYTHON}[1]{\textcolor{green}{\ttfamily #1}}

\begin{document}

\chapter{Python-Befehle}

Zur Darstellung von Python-Befehlen verwenden wir eine andere Farbe und einen anderen Font. 
Dadurch heben sich die Befehle deutlich vom Text ab. 
Beispielsweise \PYTHON{def} wird zur Definition von Funktionen verwendet. 
Dazu haben wir den eigenen Befehl {\ttfamily \textbackslash PYTHON} definiert.

\bigskip
% Darstellung von Python-Code
\renewcommand{\PYTHON}[1]{\textcolor{orange}{\ttfamily #1}}

Falls die Farbe nicht gefällt, kann man sie auch ändern. 
Mittels des  Python-Befehls \PYTHON{import} können Bibliotheken geladen werden.

\end{document}