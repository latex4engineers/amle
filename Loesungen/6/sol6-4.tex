\documentclass[12pt,a4paper]{scrbook}
\usepackage[utf8]{inputenc}
\usepackage[T1]{fontenc}
\usepackage[ngerman]{babel}

\usepackage{amsmath}
\usepackage{amsfonts}
\usepackage{amssymb}

\usepackage{graphicx}

\usepackage[style=ieee-alphabetic]{biblatex}
\addbibresource{literatur.bib}

\begin{document}
 
 
Das Buch CAGD von Gerald Farin ist ein Klassiker über Splines. \cite{Farin2001}

\bigskip

Die Norm DIN 66025 zur Programmierung von CNC-Maschinen ist ebenfalls ein Klassiker; sie behandelt allerdings keine Splines. \cite{DIN66025}

\bigskip

Herr F. Farouki hat sich sowohl mit der Programmierung von CNC-Maschinen als auch mit Splines beschäftigt. 
Sein Artikel\footnote{Mitautor ist J.~Srinathu} über einen Echtzeitinterpolator zeigt dies auch. \cite{Farouki2017}

\bigskip

Eine neue Dimension der Werkzeugmaschinen sind durch die Erfindung von 3D-Drucker entstanden. \cite{Patent3D}

\begin{quotation} 
``Optimal usage of the available resources has to be realised, while guaranteeing strict QoS requirements such as high data rates, ultra-low latency and jitter.''~\cite{Zafeiropoulos2020}
\end{quotation}
 
\printbibliography
    
\end{document}